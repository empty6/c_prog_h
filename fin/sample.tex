\documentclass[a4j,11pt]{jarticle}
\usepackage{epsfig,here}
\usepackage{url}

\setlength{\textwidth}{1.1\textwidth}
\setlength{\oddsidemargin}{-3pt}
\setlength{\evensidemargin}{\oddsidemargin}
\setlength{\topmargin}{10mm}
\setlength{\headheight}{0mm}
\setlength{\headsep}{0mm}

\begin{document}

\begin{center}
%\noindent
 \vspace{10mm}

{\bf {\huge 手続き型プログラミング基礎 期末レポート}}
%\end{center}

\vspace{80mm}

提出日:yyyy年mm月dd日

\vspace{10mm}

系/学科/類:XXXXX系

\vspace{10mm}

学籍番号:XX\_{}XXXXX

\vspace{10mm}

ログイン名:xxxxxx.xx.x

\vspace{20mm}

{\bf {\LARGE 氏名:情報 太郎}}
\end{center}

\newpage

\section{問1 単回帰分析}
\subsection*{(a) 作成したプログラムを提示し、プログラムの内容や、工夫した点について説明せよ。}
まずプログラム本体を、行番号付きで掲載すること。プログラムをそのままここにペーストしても読みづらい。
読みやすくする工夫を行うこと(listings, verbatimを利用する、など)。

説明は必ず文章で記述すること。箇条書きのみはNG(箇条書きを使ってはいけないわけではない)。

自分以外の人が読んでも次の点が明らかになり、プログラムの中身が理解できるように記述すること。

\begin{enumerate}
  \item 重要な変数、配列、関数、制御構造
  \item 用いたアルゴリズムなど
\end{enumerate}

プログラム中で工夫した点についても説明する。

\begin{enumerate}
  \item より効率の良い実装方法
  \item 手順
  \item 他の実装方法と比較した場合の処理効率の違い(演算回数、比較回数、ループの回数の比較)など
\end{enumerate}

結果が{\bf AC}にならなかった場合は、理由を考えて説明する。

改良の可能性があれば触れる。

\subsection*{(b) 回帰曲線の形を$y=ax^2+bx+c$に変更したい。どのようにプログラムを変更すべきか説明せよ。}
直接実装した結果を説明とともに記しても良いし、変更の方針だけ答えても良い。図や表などを使っても良い。


\section{問2 最長共通部分列}
\subsection*{(a) 作成したプログラムを提示し、プログラムの内容や、工夫した点について説明せよ。}
\subsection*{(b) 入力された文字列の文字数をそれぞれ$n,m$とする。プログラムの実行時間を複数計測してその結果を示し、実行時間が$n$および$m$とどのような関係にあるか推測せよ。}
複数の$n, m$について実行時間を{\tt time}コマンドで計測し、表などを用いて結果を示すこと(たとえば表\ref{time})。

\begin{table}[htbp]
  \centering
  \caption{最長共通部分列のプログラムの実行時間}
  \label{time}
  \begin{tabular}{l|rrc} \hline
  $n$ & 10 & 10 & … \\
  $m$ & 10 & 20 & … \\\hline 
  実行時間(s) & 0.002 & 0.003 & … \\\hline
  \end{tabular}
\end{table}

\subsection*{(c) 最長共通部分列が複数あるときに、重複を除いて全て出力するプログラムに変更したい。どのように変更すべきか説明せよ。}


\section{問3 ライフゲーム}
\subsection*{(a) 作成したプログラムを提示し、プログラムの内容や、工夫した点について説明せよ。}
\subsection*{(b) 格子を「周期境界」(上下端と左右端が繋がっている=トーラス状)に変更したい。どのようにプログラムを変更すべきか、説明せよ。}
周期境界等は、図\ref{life}に示すように、上下端と左右端が繋がっており、境界を超えて反対の境界に現れるような盤面である。


\begin{figure}[htbp]
  \centering
  \includegraphics{lifegame.eps}
  \caption{周期境界の説明図}
  \label{life}
\end{figure}


\subsubsection*{以下、参考文献について}
\begin{itemize}
\item 参考文献がない場合は、参考文献の章を出力しない
\item 参照していない参考文献を掲載しない
\item 参考文献の引用箇所を明示的に示す(例を参考に)
\item レポート作成時点にておいて無効なURLは参考文献にはできない
\item 書式を整えて記載する
\end{itemize}

\subsubsection*{一般的な注意事項}
作成したPDFファイルを見返して確認すること。
誤字脱字、文字化けなどがあると減点する。「、,」や「。.」が統一されていないなど、基本的なレポートの体裁を成していない場合も当然減点する。

他人のレポートからの剽窃が発覚した場合は、課題レポートの点数に加え、プログラムの点数も大幅に減点する。
他人のレポートを参考にしてはいけないわけではないが、丸写しは厳禁。
また、他人のレポートを参考にした場合は必ず出典を示すこと\cite{friend1}。
自力で考え、努力した跡が見えるプログラム・レポートを高く評価する。

\begin{thebibliography}{20}
\bibitem{Kanaoka2010} 金岡晃,島岡政基,岡本栄司.IDベース暗号の信頼構築フレームワーク, 情報処理学会論文誌,51(9),1692-1701 (2010)
\bibitem{Mirakhorli2016} M. Mirakhorli, J. Cleland-Huang. Detecting, Tracing, and Monitoring Architectural Tactics in Code, IEEE Transactions on Software Engineering, 42(3), 206-220 (2016)
\bibitem{Chiba1990} 千葉則茂, 村岡一信.レイトレーシングCG入門, Information \& Computing, vol. 46, サイエンス社 (1990)
\bibitem{KR1989} B.W. カーニハン, D.M. リッチー(石田晴久訳).プログラミング言語C 第2版 ANSI規格準拠,共立出版 (1989)
\bibitem{msdn} サンプル コードとプログラミング スタイル \url{https://msdn.microsoft.com/ja-jp/library/cc343825.aspx}(アクセス日:2017/5/18)
\bibitem{Okumura1991}奥村晴彦.C言語による最新アルゴリズム事典,技術評論社 (1991)
\bibitem{friend1} ◯◯◯◯(学籍番号 XX\_{}XXXXX,ログイン名xxxx.x.xx). 手続き型プログラミング基礎期末レポート (2017)
\end{thebibliography}
\end{document}
